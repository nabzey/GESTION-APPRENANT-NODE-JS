Comment tout ça se combine dans un projet :

TypeScript → pour écrire un code sûr et typé.

Prisma → pour accéder à la base MySQL facilement.

Zod → pour vérifier les données avant de les envoyer à la base.

MySQL → pour stocker toutes les données persistantes.

Exemple de flux complet :

Un utilisateur envoie son nom et email via une API.

Zod valide les données.

Prisma enregistre l’utilisateur dans la base MySQL.

TypeScript assure que tout est typé et sûr.

*****************
Zod est une librairie de validation et de parsing pour TypeScript.

Elle permet de vérifier que les données reçues (par exemple depuis un formulaire ou une API) sont correctes avant de les utiliser.
************

Prisma est un ORM (Object-Relational Mapping).

Il sert à interagir avec la base de données (ici MySQL) de manière plus simple et sécurisée.

Au lieu d’écrire des requêtes SQL complexes, tu peux manipuler les données avec des fonctions TypeScript.

**************
pour installer prisma :
npm install prisma @prisma/client

il y a en fait deux paquets différents :

1️⃣ prisma

C’est l’outil CLI (Command Line Interface).

Il sert uniquement au développement.

Tu l’utilises pour :

initialiser le projet (npx prisma init)

créer/modifier ton schéma (prisma/schema.prisma)

générer les migrations (npx prisma migrate dev)

générer le client (npx prisma generate)
👉 Il n’est pas utilisé dans ton code, mais plutôt dans le terminal pour préparer la base et générer le client.

2️⃣ @prisma/client

C’est la librairie que tu utilises dans ton code TypeScript/JavaScript.

Elle contient le Prisma Client généré : c’est un ensemble de classes et fonctions qui te permettent de manipuler ta base comme une API TypeScript.
👉 Sans @prisma/client, tu ne pourrais pas utiliser PrismaClient dans ton code.

Pourquoi ils sont séparés ?

Sécurité et poids : prisma (CLI) est souvent installé en dev dependency (npm install -D prisma) car tu n’en as pas besoin en production.

Clarté : @prisma/client est le code que ton appli utilise réellement, il doit donc être installé comme dépendance normale.

⚡ En résumé :

prisma → outil de dev (migration, génération du client).

@prisma/client → librairie utilisée dans ton code pour parler avec MySQL.